\begin{figure}[ht]
    \centering
    %\centerfloat
    \resizebox{0.98\textwidth}{!}{%
    \begin{tabular}{|p{1.3\textwidth}|}
%\begin{mdframed}[backgroundcolor=blue!2, font=\small]
\textbf{Study on the Structure of Index Data for \hl{c1}{Metasearch} \hl{c3}{System}} {\footnotesize(id: gakkai-e-0001384947)}

\vspace{.9em}

This paper proposes a new technique for \hl{c1}{Metasearch} \hl{c3}{system}, which is based on the grouping of both keywords and URLs.
This technique enables \hl{c1}{metasearch} \hl{c3}{systems} to \hl{c5}{share} \hl{c4}{information} and to reflect the estimation of \hl{c6}{users'} preference.
With this \hl{c3}{system}, \hl{c6}{users} can search not only by their own keywords but by similarity of HTML documents.
In this paper, we describe the principle of the grouping technique as well as the summary of the existing \textbf<2>{\hl{c2}{search} \hl{c3}{systems}}.

\vspace{1.1em}

\textbf{Mots-clés présent}: \phantom{du texte} \hl{c1}{Metasearch} --
\textbf<2>{\hl{c2}{Search} \hl{c3}{System}}

%\vspace{.7em}

\textbf{Mots-clés absent}: \\
\quad \hl{c4}{Information} \hl{c5}{Sharing} --
\hl{c4}{Information} Retrieval --
\hl{c6}{User's} Behavior -- 
Retrieval Support
%\phantom{\textbf{Mots-clés absent}:} 

\vspace{.2em}

%{\small
%\hspace{1.9cm} $\lfloor$ \hspace{.7cm} \underline{R}eordered \hspace{.6cm} $\rfloor$
%\hspace{0.1cm} $\lfloor$ \hspace{1.05cm} \underline{M}ixed \hspace{1.05cm} $\rfloor$
%\hspace{0.1cm} $\lfloor$ \hspace{.6cm} \underline{M}ixed \hspace{.6cm} $\rfloor$
%\hspace{0.1cm} $\lfloor$ \hspace{.6cm} \underline{U}nseen \hspace{.6cm} $\rfloor$
%}

%\textcolor{black}{
\hspace{.3cm}
\hspace{.7cm} \reordonne \hspace{.7cm} \hspace{0.1cm}
\hspace{1.2cm} \mixte \hspace{1.2cm} \hspace{0.1cm}
\hspace{.75cm} \mixte \hspace{.75cm} \hspace{0.1cm}
\hspace{.65cm} \nonvu \hspace{.65cm}
%}

%\end{mdframed}
\end{tabular}
}
%\caption{Exemple de document de la collection de test de NTCIR-2 (id: gakkai-e-0001384947).} %Les mots-clés auteur présent et absent ont été identifié selon la définition donnée dans ce document. %Les catégories fines (i.e.~\underline{R}éordonné, \underline{M}ixte et \underline{N}on-vu) sont explicitées.}
\label{fig:example_prmn}
\end{figure}

\iffalse
\begin{figure}[ht]
\begin{mdframed}[backgroundcolor=blue!2, font=\small]

\textbf{\hl{c3!60!white}{Rapid} \hl{c6!60!white}{Full-Text} \hl{c1!60!white}{Retrieval} Method with Coding \hl{c4!60!white}{Character} of \hl{c6!60!white}{Full-Text} Using both an Attribute and \hl{c4}{Character} \hl{c5}{Location}}

\vspace{.9em}

This paper describes \hl{c3}{rapid} \hl{c6}{full-text} \hl{c1}{retrieval} method with software and the result of retrieval-experiment. Generally in Japanese document,same Kanji \hl{c4!60!white}{character} rarely appeares and also same Kanji \hl{c2}{string} rarely appeares. This method uses these characteristics of Japanese document. We are able to rapidly \hl{c1}{retrieval} \hl{c6}{full-text} with this method.

\say{\texttt{[...] and \hl{c4}{Character} \hl{c5}{Location}}}
\say{\texttt{[...] rapidly \hl{c1}{retrieval} \hl{c6}{full-text} with [...]}}
\say{\texttt{[...]  [...]}}
\say{\texttt{[...]  [...]}}


\vspace{1.1em}

\textbf{Mots-clés présent}:
\hl{c6}{Full-text}~\hl{c1}{Retrieval} -- \hl{c4}{Character}~\hl{c5}{Location}

\vspace{.7em}

\textbf{Mots-clés absent}:
\hl{c4}{Character}~\hl{c2}{String} -- \hl{c3}{Rapid}~\hl{c1}{Retrieval} -- Information~\hl{c1}{Retrieval} -- \\
\phantom{\textbf{Mots-clés absent}:} Character-string~Collation



\end{mdframed}
\caption{Exemple de document de la collection de test de NTCIR-2 (id: gakkai-e-0000056172).} %Les mots-clés auteur présent et absent ont été identifié selon la définition donnée dans ce document. %Les catégories fines (i.e.~\underline{R}éordonné, \underline{M}ixte et \underline{N}on-vu) sont explicitées.}
\label{fig:example_prmn}
\end{figure}
\fi