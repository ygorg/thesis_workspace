\section{Contribution: Validation des méthodes génératives}

\subsection{Constitution du jeu de données}

\begin{frame}{Motivation}
    
    \begin{itemize}
        \item Actuellement \textbf{un seul} jeu de données de grande taille.
        \begin{itemize}
            \item Insuffisant pour obtenir des conclusions fiables. %Cette seule évaluation ne permet pas d'avoir des conclusions fiables.
            \item Résultats \textbf{transposables} à d'autres jeux de données?
        \end{itemize}
        
        \item Nécessité de construire un nouveau jeu de données avec:
        \begin{itemize}
            \item documents \textbf{annotés en mots-clés};
            \item \textbf{suffisamment} de documents pour entraîner des méthodes neuronales;
            \item \textbf{annotation différente} de l'annotation auteur.
        \end{itemize}
    \end{itemize}
    
    \pause
    
    \begin{itemize}
        \item Les \textbf{articles journalistiques} sont disponibles en \textbf{grande quantité} sur internet
        \item Sont souvent \textbf{annotés en mots-clés} pour le référencement
    \end{itemize}
    
\end{frame}
\begin{frame}{Sources de données}
    \begin{columns}
    \begin{column}{.45\textwidth}
    \textbf{NewYork Times}
    \begin{itemize}
        \item Annotation éditeur
        %\item Période: 2006 -- 2017
        \item $\Rightarrow$ 296\,974 articles
    \end{itemize}
    \end{column}
    \begin{column}{.54\textwidth}
    \textbf{Japan Times}
    \begin{itemize}
        \item \'Evaluer la généralisation
        %\item Période: 2008 -- 2019
        \item $\Rightarrow$ 11\,057 articles
    \end{itemize}
    \end{column}
    \end{columns}

    \vspace{.5cm}
    \begin{itemize}
        \item Filtrage des documents trop longs, trop courts et redondants.
    \end{itemize}

%\end{frame}

%\begin{frame}{Statistiques}
    
    \begin{table}[htbp!]
\centering
\resizebox{0.98\textwidth}{!}{
    \begin{tabular}{lccrrrrrr}
     & & \multicolumn{4}{c}{Corpus} & \multicolumn{3}{c}{Document} \\% & Mots-clés \\
    \cmidrule(lr){4-6} \cmidrule(lr){7-9} \\[-1.2em]% \cmidrule(lr){9-9} \\[-1.2em]
    \textbf{Corpus} & \textbf{Ann.} & \textbf{Lang.} & \textbf{\#Entr.} & \textbf{\#Val.} & \textbf{\#Test} & \textbf{\#mots} & \textbf{\#mc} & \textbf{\%abs} \\%& \textbf{\#uniq.} \\
    \cmidrule(lr){1-3} \cmidrule(lr){4-6} \cmidrule(lr){7-9}
    KPTimes       & $E$ &en&260\,K&20\,K& 20\,K & 738 &   5,0 & 38,4 \\%& 20\,535 \\
    \quad+ NYTimes & $E$&en&260\,K&20\,K& 10\,K & \cb<2>{color1!40}{905} &   5,0 & \cb<2>{color1!40}{52,5} \\%& 13\,387 \\
    \quad+ JPTimes & $E$&en&    - &   - & 10\,K & \cb<2>{color1!40}{570} &   5,0 & \cb<2>{color1!40}{24,2} \\%&  8\,611 \\
    \bottomrule
    \end{tabular}
}
\end{table}
    
\end{frame}


%\begin{frame}{Exemple de document du NewYork Times}
%    \begin{figure}[!htbp]
\begin{minted}[
frame=lines,framesep=2mm,
baselinestretch=0.9,fontsize=\scriptsize
]{html}
<html><body>
<title>Muslim Women in Hijab Break Barriers: ‘Take the Good
With the Bad’ - The New York Times</title>
<meta name="description" content="
    Even as reports of hate crimes against Muslims rise in
    America and Canada, hijabis are appearing in makeup ads, 
    beauty pageants and news anchor chairs.
    ">
<meta name="byl" content="By Katie Rogers">
<meta name="news_keywords" content="
    Muslim Veiling, Hate crime, Women and Girls, Canada, US,
    Islam, Fashion,News media;journalism">
<meta name="CG" content="world">
<meta name="SCG" content="americas">
<meta name="pdate" content="20161208">
<meta name="url" content="
    https://www.nytimes.com/2016/12/08/
    world/americas/hijab-muslim-women.html">
...
</head><body>
When Ginella Massa, a Toronto-based ...
</body></html>
\end{minted}
    \caption{Version \texttt{html} d'un article du jeu de données KPTimes.}
\end{figure}
%\end{frame}

\begin{frame}{Processus d'annotation éditeur}

    \begin{figure}
        \centering

\begin{tikzpicture}[minimum width=1.5cm, minimum height=1cm, scale=.7,transform shape]
    \tikzstyle{edge label}=[above=-.25cm,sloped,scale=1,black!60]
    \tikzstyle{label}=[scale=.75, text width=2cm, align=center]
    
    \pic[local bounding box=doc] (doc) {doc={scale 1.5}};
    
    \node[below=0.5cm of doc, rounded corners, text width=2.5cm, align=center, draw=color0!80!black, fill=color0!30] (auto) {\footnotesize Extraction automatique de mots-clés};
    \pic[below=1.5cm of auto,local bounding box=kwsauto] {kws={scale 1.5}};
    \node[right=of auto, rounded corners, draw=color0!80!black, fill=color0!30] (editor) {Editeurs};

    \begin{scope}[local bounding box=kwsadd, scale=1.5]
    \def\xdist{1.5-.325}
    \def\ydist{1-.4}
    \draw[thick] ($(editor)+(\xdist,\ydist)$) rectangle +(.65,.8);
    \fill[color3] ($(editor)+(\xdist,\ydist)$) ++(.125,.5-.05) rectangle +(.25,.1);
    \fill[color4] ($(editor)+(\xdist,\ydist)$) ++(.125,.3-.05) rectangle +(.35,.1);
    \end{scope}
    
    \begin{scope}[local bounding box=kwsfilt, scale=1.5]
    \def\xdist{1.5-.325}
    \def\ydist{-1-.4}
    \draw[thick] ($(editor)+(\xdist,\ydist)$) rectangle +(.65,.8);
    \fill[color1] ($(editor)+(\xdist,\ydist)$) ++(.125,.5-.05) rectangle +(.3,.1);
    \fill[color2] ($(editor)+(\xdist,\ydist)$) ++(.125,.3-.05) rectangle +(.2,.1);
    \end{scope}

    \begin{scope}[local bounding box=kws, scale=1.5]
    \def\xdist{3-.325}
    \def\ydist{0-.6}
    \draw[thick] ($(editor)+(\xdist,\ydist)$) rectangle +(.65,1.2);
    \fill[color3] ($(editor)+(\xdist,\ydist)$) ++(.125,.9-.05) rectangle +(.25,.1);
    \fill[color4] ($(editor)+(\xdist,\ydist)$) ++(.125,.7-.05) rectangle +(.35,.1);
    \fill[color1] ($(editor)+(\xdist,\ydist)$) ++(.125,.5-.05) rectangle +(.3,.1);
    \fill[color2] ($(editor)+(\xdist,\ydist)$) ++(.125,.3-.05) rectangle +(.2,.1);
    \end{scope}
    
    \draw[->] (doc) -- (auto);
    \draw[->] (auto) -- (kwsauto);
    \draw[->] (kwsauto) to[out=0, in=180] (editor);
    \draw[->] (doc) to[out=0, in=180] (editor);
    \draw[->] (editor) to[out=0, in=180] node [edge label] {Ajout} (kwsadd);
    \draw[->] (editor) to[out=0, in=180] node [edge label] {Filtre} (kwsfilt);
    
    \draw[->] (kwsadd) to[out=0, in=180] (kws);
    \draw[->] (kwsfilt) to[out=0, in=180] (kws);
    
\end{tikzpicture}
    \end{figure}
    
    \begin{itemize}
        \item Annotation \textbf{semi-automatique} basée sur un vocabulaire contrôlé
        \item Les éditeurs \textbf{valident} et \textbf{complètent} les mots-clés proposés
        \item Annotation \textbf{cohérente} (vocabulaire contrôlé) et \textbf{exhaustive} (ajout de mots-clés)
    \end{itemize}
    
%   \begin{table}[]
%       \centering
        %ny0048697
%        \begin{tabular}{cp{3.5cm}p{2cm}p{1.5cm}}
%Catégories & Acteurs & Lieux & Sujets \\
%\midrule
%Human Rights &  Ilham~Tohti; Communist~Party of~China & Urumqi~China; Beijing & Censorship; Uyghur \\
%        \end{tabular}
%    \end{table}
\end{frame}

\begin{frame}{Cohérence de l'annotation}
    \usepgfplotslibrary{fillbetween}

\begin{figure}[ht!]
    \centering
    %\resizebox{0.7\textwidth}{!}{%
    \begin{tikzpicture}
	    \begin{axis}[height=5cm,
	                 width=8.5cm,
	                 grid style={dashed,gray!30},
	                 ymajorgrids,
                     xlabel={Nombre d'assignements},
                     ylabel={\% de MC de référence},
                     every node near coord/.append style={font=\footnotesize},
                     ybar=0pt,
                     %ybar,
                     bar width=9pt,
                     ymin=0,
                     ymax=100,
                     xmin=0.3,
                     xmax=5.7,
                     tick align=inside,
                     legend entries={
                        %KPCrowd (lecteurs),
                        Auteurs (KP20k),
                        \'Editeurs (KPTimes)
                    },
                     legend cell align={left},
                     legend style={font=\small},
                     ]

            % 500N-KPCrowd Test
            %\addplot [draw=green!80!black, fill=green!10, pattern=north east lines] coordinates {
            %    (1, 86.8)(2, 9.0)(3, 2.9)(4, 1.0)(5, 0.1)
            %};
            
            % DUC-2001 Test
            %\addplot [draw=green, fill=green!15] coordinates {
            %    (1, 82.1)(2, 9.6)(3, 4.2)(4, 1.5)(5, 0.8)
            %};

            % KP20k Test
            \addplot [draw=blue, fill=blue!15] coordinates {
                (1, 79.4)(2, 10.2)(3, 3.5)(4, 1.8)(5, 1.1)
            };
            
            % KPTimes
            \addplot [draw=red, fill=red!15, pattern=dots, nodes near coords, point meta=explicit symbolic] coordinates {
                (1, 64.2)(2, 12.3)(3, 5.6)(4, 3.2)(5, 2.1)
            };
            
            \node (A1) at (axis cs:0.73,79.4) {};
            \node (A2) at (axis cs:1.26,64.4) {};
            \draw[->,>=latex,red!80!black] (A1) to[bend left] node (sum) [midway, right, font=\small] %{\textcolor{red!80!black}{-22,6}} (A2);
            {\textcolor{red!80!black}{-15,2}} (A2);
            
            \node (B1) at (axis cs:1.73,9) {};
            \node (B2) at (axis cs:2.26,12.4) {};
            \draw[->,>=latex,green!60!black] (B1) to[bend left] node (sum) [midway, above, font=\small] %{\textcolor{green!60!black}{+3,3}} (B2);
            {\textcolor{green!60!black}{+2,1}} (B2);
            
            \node (C1) at (axis cs:2.73,2.9) {};
            \node (C2) at (axis cs:3.26,6.1) {};
            \draw[->,>=latex,green!60!black] (C1) to[bend left] node (sum) [midway, above, font=\small] %{\textcolor{green!60!black}{+2.7}} (C2);
            {\textcolor{green!60!black}{+2,1}} (C2);
            
            \node (D1) at (axis cs:3.73,1.0) {};
            \node (D2) at (axis cs:4.26,3.0) {};
            \draw[->,>=latex,green!60!black] (D1) to[bend left] node (sum) [midway, above, font=\small] %{\textcolor{green!60!black}{+2,2}} (D2);
            {\textcolor{green!60!black}{+1,4}} (D2);
            
            \node (E1) at (axis cs:4.73,0.1) {};
            \node (E2) at (axis cs:5.26,2.0) {};
            \draw[->,>=latex,green!60!black] (E1) to[bend left] node (sum) [midway, above, font=\small] %{\textcolor{green!60!black}{+2,0}} (E2);
            {\textcolor{green!60!black}{+1,0}} (E2);

        \end{axis}
    \end{tikzpicture}%
    %}
    %\vspace*{-1em}
    %\caption{Distributions de l'assignement des mots-clés de référence.}
    %\label{fig:kptimes_kw_distrib}
    %\vspace*{-1em}
\end{figure}


% KP20K Train
%\addplot [draw=blue, fill=blue!15] coordinates {
%    (1, 69.7)
%    (2, 13.1)
%    (3, 5.0)
%    (4, 2.7)
%    (5, 1.7)
%};

%TermITH-Eval
%\addplot [draw=red, fill=red!15, nodes near coords, point meta=explicit symbolic] coordinates {
%    (1, 68.2) [\textcolor{red!80!black}{~~~~~-11.2}]
%    (2, 16.4) [\textcolor{green!60!black}{+6.2}]
%    (3, 5.4) [\textcolor{green!60!black}{+1.9}]
%    (4, 3.4) [\textcolor{green!60!black}{+1.6}]
%    (5, 1.9) [\textcolor{green!60!black}{+0.8}]
%};

% DUC-2001 Test
%\addplot [draw=green, fill=green!15] coordinates {
%    (1, 82.1)
%    (2, 9.6)
%    (3, 4.2)
%    (4, 1.5)
%    (5, 0.8)
%};

% KDD
%\addplot [draw=blue, fill=blue!15] coordinates {
%    (1, 78.6)
%    (2, 11.5)
%    (3, 3.6)
%    (4, 1.6)
%    (5, 1.1)
%};
    \begin{itemize}
        \item 80\% de mots-clés associés à un seul document pour l'annotation auteur
        %\item Il y a moins d'hapax dans KPTimes que dans KP20k.
    \end{itemize}
    
    \textbf{Hypothèses}:
    \begin{itemize}
        \item \'Evaluation plus fiable
        \item Apprentissage plus efficace des méthodes génératives
    \end{itemize}
\end{frame}


\subsection{Cadre expérimental}

\begin{frame}{Cadre Expérimental}
    \begin{block}{Méthodes extractives}
    \begin{itemize}
        \item \tfidf{}~\cite{jones_statistical_1972}: spécificité des mots
        \item MultiPartiteRank~\cite{boudin_unsupervised_2018}: centralité des mots
        \item Kea~\cite{witten_kea:_1999}: classifieur bayesien
    \end{itemize}
    \end{block}
    
    \begin{block}{Méthodes génératives}
    \begin{itemize}
        \item CopyRNN
        \begin{itemize}
        \item \colorbox{color2!40}{CopyNews}: Entraîné sur \colorbox{color2!40}{KPTimes} (articles journalistiques)
        \item \colorbox{color1!40}{CopySci}: Entraîné sur \colorbox{color1!40}{KP20k} (notices scientifiques)
        \end{itemize}
    \end{itemize}
    \end{block}
    
    Utilisation des paramètres recommandés par les auteurs.
    
    \textbf{Métrique}: F-mesure sur les 10 meilleurs mots-clés
\end{frame}

\begin{frame}<1,4>{Cadre Expérimental}
    \begin{table}[htbp!]
\centering
\resizebox{0.98\textwidth}{!}{
    \begin{tabular}{lcrrrr
    S[table-format=2.1,table-number-alignment=right] S[table-format=2.1,table-number-alignment=right] S[table-format=5.0,table-number-alignment=right] S[table-format=1.1,table-number-alignment=right]}
    
     & & \multicolumn{3}{c}{Corpus} & 
    \multicolumn{3}{c}{Document} & 
    \multicolumn{2}{c}{Mots-clés} \\
    \cmidrule(lr){3-5} \cmidrule(lr){6-8} \cmidrule(lr){9-10} \\[-1.2em]
    
    \textbf{Corpus} &
    \textbf{Ann.} &
    \textbf{\#Entr.} &
    \textbf{\#Val.} &
    \textbf{\#Test} &

    \textbf{\#mots} &
    \textbf{\#mc} &
    \textbf{\%abs} &
    
    \textbf{\#uniq.} &
    \textbf{\#ass.} \\

    \cmidrule(lr){1-2} \cmidrule(lr){3-5} \cmidrule(lr){6-8} \cmidrule(lr){9-10}

    \textbf{KPTimes}       & $E$ &260\,K&20\,K &  20\,K & 738 &   5.0 & 38.4 & 20535 & 5.0 \\ % 1.03
    \quad \textbf{JPTimes} & $E$ &    - &     - &  10\,K & 570 &   5.0 & 24.2 &  8611 & 5.9 \\ % 0.86
    \quad \textbf{NYTimes} & $E$ &260\,K&20\,K &  10\,K & 905 &   5.0 & 52.5 & 13387 & 3.8 \\ % 1.34
    \addlinespace
    KPCrowd       & $L$ &  450 &     - &     50 & 465 &  46.2 &  8.1 &  1937 & 1.2 \\ % 38.74
    DUC-2001      & $L$ &    - &     - &    308 & 847 &   8.1 &  3.1 &  1800 & 1.4 \\ % 5.84
    \addlinespace
    KP20k         & $A$ &530\,K& 20\,K &  20\,K & 176 &   5.3 & 42.4 & 53489 & 2.0 \\ % 2.67

    \bottomrule

    \end{tabular}
}
\caption{Comparaison des statistiques de KPTimes et ses sous-ensembles de test JPTimes et NYTimes avec les jeux de données d'articles journalistiques et KP20k. Les mots-clés de référence sont annotés par des \underline{$L$}ecteurs, des \underline{$E$}diteurs ou des \underline{$A$}uteurs. La table présente le nombre de documents dans les corpus d'entraînement (\#Entr.), de validation (\#Val.) et de test (\#Test) ainsi que le nombre moyen de mots (\#mots), de mots-clés (\#mc) et le ratio de mots-clés absents (\%abs) par document. Les colonnes \#uniq. et \#ass. montrent le nombre de mots-clés uniques et le nombre moyen d'assignation d'un mot-clé.}
\label{tab:kptimes_stats}
\end{table}
    \begin{itemize}
        %\item Les documents de JPTimes sont plus courts que ceux de NYTimes et DUC-2001.
        %\item JPTimes contient moitié moins de mots-clés absents que NYTimes.
        \item Annotation lecteur de DUC-2001: plus de mots-clés que les autres jeux de données et majoritairement présents.
    \end{itemize}
    
\end{frame}

\subsection{Résultats}

\begin{frame}{Document similaire, type d'annotation similaire}

La supériorité de CopyRNN est-elle toujours présente avec NYTimes ?
%Les résultats obtenus sur KP20k sont-ils répétables sur NYTimes ?

\begin{table}[htbp!]
\centering
%\resizebox{\textwidth}{!}{
\begin{tabular}{lc}
    \small{$\text{F}@10$} & \colorbox{color2!40}{\textbf{NYTimes}} \\
    \midrule
    \tfidf{}      & \pad{0}9,6 \\
    MPRank        & 11,2 \\
    Kea           & 11,0 \\
    \addlinespace
    \colorbox{color2!40}{CopyNews} & \best{39,3}\\
    \bottomrule
\end{tabular}
%}
\end{table}
    
    \begin{itemize}
        \item Comme sur KP20k, CopyRNN obtient toujours de meilleurs résultats que les méthodes extractives.
    \end{itemize}
\end{frame}

\begin{frame}<1,2,3,5>[label=kptimesabs]{Document similaire, type d'annotation différent}

Les performances de CopyRNN sont-elles généralisables à un type d'annotation différent ?

\begin{table}[htbp!]
\centering
\begin{tabular}{lccc}
    & \'Editeur & \cb<2>{color1!40}{\'Editeur} & \cb<3>{color1!40}{Lecteur} \\
    \small{$\text{F}@10$} & \colorbox{color2!40}{\textbf{NYTimes}} & \textbf{JPTimes} & \textbf{DUC-2001} \\
    \midrule
    \tfidf{}      & \pad{0}9,6 & 15,1 & 23,0 \\
    MPRank        & 11,2 & 16,8 & 25,3\\
    Kea           & 11,0 & 16,6 & \best{26,2}\\
    \addlinespace
    \colorbox{color2!40}{CopyNews} & \cb<2,3>{color1!40}{\best{39,3}} & \cb<2>{color1!40}{\best{24,6}} & \cb<3,5>{color1!40}{10,5} \\
    \bottomrule
    \only<4>{\small \%abs. & \small 52,5 & \small 24,2 & \small 3,1\\}
\end{tabular}
\end{table}
    
    \only<1-4>{
    \begin{itemize}
        \item CopyNews connaît une première baisse de performances lors de l'évaluation sur JPTimes.
        \item CopyNews généralise mal à un type d'annotation différent.
        \only<4>{\item Méthodes extractives désavantagés par le taux de mots-clés absent}
        \end{itemize}
    }
    
    \only<5>{
    \begin{block}{Exemple de mots-clés d'un article de DUC-2001 (\footnotesize AP890511-0126):}
    \hspace{.15cm}\footnotesize
    \colorbox{color2!40}{\textbf{CopyNews}}: \textcolor{color0}{tuberculosis} -- \textcolor{color3}{us} -- \textcolor{color9}{prisons} -- new~jersey --  medicine~and~health

    \textbf{M.-c. lecteur}: \textcolor{color0}{tuberculosis}~rate -- \textcolor{color3}{u.s}.~\textcolor{color9}{prisons} -- aids-virus~infections --\\
    \phantom{\textbf{M.-c. lecteur}:} \textcolor{color0}{tuberculosis}~cases -- airborne transmission --  cdc
    \end{block}
    }
    
    % \begin{block}{Exemple de mots-clés d'un article de JPTimes ({\footnotesize jp0007430}):}
    % \hspace{.15cm}\footnotesize
    % \colorbox{color2!40}{\textbf{CopyNews}}: \textcolor{color0}{oxfam} -- economy -- \textcolor{color9}{davos} world economic forum -- \textcolor{color3}{poverty} -- \underline{us economy}

    % \textbf{M.-c. lecteur}: \textcolor{color0}{oxfam} -- wealth -- \textcolor{color9}{davos} -- jeff bezos -- \textcolor{color3}{poverty}
    % \end{block}
\end{frame}   

%\begin{frame}{Document similaire, annotation différente (exemple)}
%    \begin{figure}
    \centering
    %\centerfloat
    \resizebox{0.98\textwidth}{!}{%
    \begin{tabular}{|p{1.3\textwidth}|}
    
    The \textcolor{color4}{tuberculosis} \textcolor{color8}{rate} in \textcolor{color7}{U.S.} \textcolor{color6}{prisons} may be more than three times higher than on the outside, federal \textcolor{color2}{health} officials said Thursday, urging testing, isolation and other measures to curb TB behind bars.
    %Researchers with the Centers for Disease Control cited a survey in 29 states, where \textcolor{color6}{prisons} reported 31 \textcolor{color4}{tuberculosis} \textcolor{color1}{cases} per 100,000 inmates in 1984-85, compared with eight \textcolor{color1}{cases} per 100,000 reported among non-incarcerated adults in those states during the same period.
    %\say{In some large correctional systems, the incidence of TB has increased dramatically,} the \textcolor{color5}{CDC} said, noting that in New York state there were 106 TB \textcolor{color1}{cases} per 100,000 inmates in 1986 -- seven times more than the average of 15 \textcolor{color1}{cases} reported in 1976-78.
    %In \textcolor{color2}{New Jersey}, inmates had a TB \textcolor{color8}{rate} of 110 per 100,000 in 1987, 11 times higher than the general \textcolor{color2}{New Jersey} population. In California, the \textcolor{color8}{rate} was nearly six times higher -- 80 per 100,000. \textcolor{color4}{Tuberculosis}, a contagious, bacterial lung disease, occurs in about 22,000 new \textcolor{color1}{cases} each year in the United States ; most can be cured with drug treatment.
    %As many as 7 percent of Americans have latent TB \textcolor{color0}{infections}, and about 10 percent of them will someday develop a \textcolor{color1}{case} of \textcolor{color4}{tuberculosis} itself.
    [\ldots]
    \say{Persons at highest risk... are close contacts,} the \textcolor{color5}{CDC} said, noting that TB can pose particular problems in \textcolor{color6}{prisons}, where there is often overcrowding and \say{where the environment is often conducive to \textcolor{color9}{airborne transmission} of \textcolor{color0}{infection} among inmates, staff and visitors.}
    %The \textcolor{color5}{CDC}'s Advisory Committee for the Elimination of \textcolor{color4}{Tuberculosis} is recommending TB testing for most new \textcolor{color6}{prison} inmates and staff members -- with the possible exception of inmates just transferring through for less than a week.
    %The \textcolor{color5}{CDC} committee also recommends new tests at least once a year, rapid chest X-rays for TB-infected people showing symptoms, and isolation -- off the \textcolor{color6}{prison} property, if necessary -- for those with suspected or confirmed symptomatic TB \textcolor{color1}{cases}.
    [\ldots]
    The spread of \textcolor{color3}{AIDS-virus} \textcolor{color0}{infections} may play a part in the spread of TB in \textcolor{color6}{prisons}, the \textcolor{color5}{CDC} said.
    AIDS weakens the immune system, making patients susceptible to \textcolor{color0}{infections} other people might ward off, including \textcolor{color4}{tuberculosis}.
    AIDS tests should be offered to all inmates with known TB \textcolor{color0}{infections}, the \textcolor{color5}{CDC} report said.

    \vspace{1.1em}
    
    \textbf{CopyNews}: \textcolor{color4}{tuberculosis}\phantom{~rate} -- \textcolor{color6}{prisons} -- \textcolor{color7}{us} -- \textcolor{color2}{new~jersey} --  medicine~and~\textcolor{color2}{health} \\
    %
    \textbf{Réf. lecteur}: \textcolor{color4}{tuberculosis}~\textcolor{color8}{rate} -- \textcolor{color7}{u.s}.~\textcolor{color6}{prisons} -- \textcolor{color3}{aids-virus}~\textcolor{color0}{infections} --
    \\
    \phantom{\textbf{Réf. lecteur}:}
    \textcolor{color4}{tuberculosis}~\textcolor{color1}{cases} --
    \textcolor{color9}{airborne transmission} --  \textcolor{color5}{cdc}
    \vspace{0.2em}


    \end{tabular}%
    }
    %\caption{Exemple de document de l'ensemble de test du corpus DUC-2001 (id: {\small AP890511-0126})).}
    \label{fig:ex_kptimes_duc}
\end{figure}

%\end{frame}

\begin{frame}<1,2>{Document différent, type d'annotation différent}
    Les performances de CopyRNN sont-elles généralisables à d'autres genres de documents ?

    \begin{table}[htbp!]
    \centering
    %\resizebox{\textwidth}{!}{
\begin{tabular}{l c c}
  & \'Editeur & Auteur \\
\small{$\text{F}@10$}  & \cellcolor{color2!40} \textbf{KPTimes} & \cellcolor{color1!40} \textbf{KP20k} \\
\midrule
\cellcolor{color2!40} CopyNews &\cellcolor{color2!40}\best{31,9}&\pad{0}6,6 \\
\cellcolor{color1!40} CopySci  &    14,9 &\cellcolor{color1!40}\best{25,5} \\
\bottomrule
\end{tabular}
    %}
    %\caption{Performances de généralisation du modèle CopyRNN entraîné sur KPTimes et KP20k.}
    \label{tab:kptimes_perf_generalisation}
\end{table}
    
    \only<1>{
    \begin{itemize}
    \item CopyNews obtient de meilleures performances grâce à son annotation plus cohérente.
    \item Faible généralisation à un type d'annotation et un genre différent.
    \end{itemize}
    }
    
    \only<2-3>{
    \begin{block}{Exemple de mots-clés d'un article de \textbf{KP20k} ({\small 011355}):}
    \hspace{.15cm}\footnotesize
    \colorbox{color2!40}{\textbf{CopyNews}}: research -- science~and~technology -- medicine~and~health -- \\ \phantom{\textbf{CopyNews}:} \textbf<3>{diagnostic~problem-solving} -- science~journal
    
    \colorbox{color1!40}{\textbf{M.-c. auteur}}: \textbf<3>{diagnosis} -- \textbf<3>{multiple~disorders} -- \textbf<3>{competition} -- \textbf<3>{neural~networks} --\\
    \phantom{\textbf{M.-c. auteur}:} \textbf<3>{learning}
    \end{block}
    }
    
    %\begin{block}{Exemple de mots-clés d'un article de KPTimes:}
    %\colorbox{color1!40}{\textbf{CopySci}}: \\
    %\textbf{M.-c. éditeur}: 
    %\end{block}
\end{frame}


%\begin{frame}{Document différent, annotation différente (exemple)}
%    \begin{figure}
    \centering
    %\centerfloat
    \resizebox{0.98\textwidth}{!}{%
    \begin{tabular}{|p{1.3\textwidth}|}
    \textbf{\textcolor{color7}{Multiple} \textcolor{color0}{disorder} \textcolor{color1}{diagnosis} with adaptative \textcolor{color6}{competitive} \textcolor{color5}{neural} \textcolor{color8}{networks}}

    \vspace{0.9em}

    Backpropagation \textcolor{color5}{neural} \textcolor{color8}{networks} have repeatedly been used for \textcolor{color4}{diagnostic} \textcolor{color3}{problem-solving}, but have not been demonstrated to work well when \textcolor{color7}{multiple} \textcolor{color0}{disorders} are present.
    We hypothesized that letting nodes in a backpropagation \textcolor{color5}{neural} \textcolor{color8}{network} compete to be part of a \textcolor{color4}{diagnostic} solution would produce better performance than the use of existing backpropagation methods.
    %To test this hypothesis, we derived an error backpropagation \textcolor{color2}{learning} rule that can be used with \textcolor{color6}{competitive} units (\textcolor{color6}{competitive} backpropagation).
    %Artificial \textcolor{color5}{neural} \textcolor{color8}{networks} were then trained using both this new \textcolor{color2}{learning} rule and standard error backpropagation on a specific medical \textcolor{color1}{diagnosis} problem: identification of the location of damage in the brain given a set of examination findings.
    [...]
    %Training samples included solely 'prototypical' cases where a single location of damage is present.
    %The trained \textcolor{color8}{networks} were then tested with atypical cases where the manifestations of more than one \textcolor{color0}{disorder} were present or only a single manifestation was present.
    %\textcolor{color8}{Networks} employing \textcolor{color6}{competition} among units were found to perform qualitatively better with these multiple-disorder cases than standard \textcolor{color8}{networks} and also to perform better on single-manifestation cases.
    %The reasons for this are explained.
    %The \textcolor{color6}{competitive} backpropagation \textcolor{color2}{learning} rule described here provides a promising new tool for adaptive \textcolor{color4}{diagnostic} \textcolor{color3}{problem-solving}.

    \vspace{1.1em}

    \textbf{CopyRNN}: \textcolor{color5}{competitive}~\textcolor{color1}{backpropagation}, \textcolor{color1}{backpropagation}, \textcolor{color1}{adaptive}, \textcolor{color8}{neural}~\textcolor{color2}{networks}, \textcolor{color5}{competitive}~\textcolor{color0}{learning}\\

    \textbf{CopyNews}: research, science~and~technology, medicine~and~health, \textcolor{color4}{diagnostic}~\textcolor{color3}{problem-solving}, science~journal\\
    \textbf{Réf. auteur}: \textcolor{color1}{diagnosis}, \textcolor{color7}{multiple}~\textcolor{color0}{disorders}, \textcolor{color6}{competition}, \textcolor{color5}{neural}~\textcolor{color8}{networks}, \textcolor{color2}{learning}
    \vspace{0.2em}

    \end{tabular}%
    }
    %\caption{Exemple de document de l'ensemble de test du corpus KP20k (id: {\small 011355}))}%. Les mots composants termes-clés présent dans le document sont soulignés.}
\end{figure}


\iffalse
\begin{figure}
    \centering
    %\centerfloat
    \resizebox{0.98\textwidth}{!}{%
    \begin{tabular}{|p{1.3\textwidth}|}
    \textbf{Ex-Nissan chief Carlos Ghosn to be served with fresh arrest warrant }

    \vspace{0.9em}

    Tokyo prosecutors have decided to seek a fresh arrest warrant for former Nissan Motor Co. Chairman Carlos Ghosn on suspicion that he failed to report around ¥4 billion (\$35.5 million) of his remuneration in its securities reports for the three years through March, sources close to the matter said Tuesday.
    %Ghosn, who is being held at the Tokyo Detention House, has already been accused of breaching the Financial Instruments and Exchange Act after allegedly reporting only ¥5 billion of his ¥10 billion compensation during the five years through March 2015.
    %Along with Greg Kelly, a former Nissan representative director, Ghosn is expected to be served with a fresh arrest warrant on Dec. 10, when the detention period for the pair expires.
    %Ghosn and Kelly, who was arrested along with the former chairman on Nov. 19 for alleged conspiracy, have told the prosecutors that it was unnecessary to report some of the remuneration as the payments had yet to be settled, according to different sources with knowledge of the investigation.
    %The unreported remuneration of the 64-year-old charismatic automotive industry figure, known for rescuing Nissan from the brink of bankruptcy in the 1990s, is believed to total ¥9 billion.
    %In Japan, crime suspects can be kept in custody for 10 days and that can be extended for another 10 days if a judge grants prosecutors’ request for extension.
    %At the end of that period, prosecutors must file a former charge or let the suspect go.
    %However, they can also arrest suspects for a separate crime, in which case the process starts over again.
    %This process can be repeated, sometimes keeping suspects detained for months without formal charges and without bail.

    \vspace{1.1em}

    \textbf{CopyRNN}: fresh arrest warrant, ghosn, fresh arrest, ex-nissan, bankruptcy\\
    \textbf{CopyNews}: ghosn carlos, carlos ghosn, japan, tokyo detention house, billion compensation\\
    \textbf{Mots-clés de référence}: arrest, nissan, mitsubishi motors, renault, carlos ghosn
    \vspace{0.2em}

    \end{tabular}%
    }
    %\caption{Exemple de document de l'ensemble de test du corpus KPTimes (id: {\small jp0010292})).}
    \label{fig:_}
\end{figure}





\begin{figure}
    \centering
    %\centerfloat
    \resizebox{0.98\textwidth}{!}{%
    \begin{tabular}{|p{1.3\textwidth}|}
    \textbf{Exploring a \textcolor{color6}{digital} \textcolor{color5}{library} through \textcolor{color8}{key} \textcolor{color2}{ideas}.}

    \vspace{0.9em}

    \textcolor{color8}{Key} \textcolor{color2}{Ideas} is a technique for exploring \textcolor{color6}{digital} \textcolor{color5}{libraries} by navigating passages that repeat across multiple \textcolor{color3}{books}.
    From these popular passages emerge \textcolor{color1}{quotations} that authors have copied from \textcolor{color3}{book} to \textcolor{color3}{book} because they capture an \textcolor{color2}{idea} particularly well: \textcolor{color0}{Jefferson} on liberty; \textcolor{color4}{Stanton} on women's rights; and Gibson on cyberpunk.
    We augment Popular Passages by extracting \textcolor{color8}{key} terms from the surrounding context and computing sets of related \textcolor{color8}{key} terms.
    We then create an interaction model where readers fluidly explore the \textcolor{color5}{library} by viewing popular \textcolor{color1}{quotations} on a particular \textcolor{color8}{key} term, and follow links to \textcolor{color1}{quotations} on related \textcolor{color8}{key} terms.
    %In this paper we describe our vision and motivation for \textcolor{color8}{Key} \textcolor{color2}{Ideas}, present an implementation running over a massive, real-world \textcolor{color6}{digital} \textcolor{color5}{library} consisting of over a million scanned \textcolor{color3}{books}, and describe some of the technical and design challenges. The principal contribution of this paper is the interaction model and prototype system for browsing \textcolor{color6}{digital} \textcolor{color5}{libraries} of \textcolor{color3}{books} using \textcolor{color8}{key} terms extracted from the aggregate context of popularly quoted passages.

    \vspace{1.1em}

    \textbf{CopyNews}: \textcolor{color3}{books}, \textcolor{color4}{stanton}, \textcolor{color0}{jefferson}, art, \textcolor{color5}{library}\\
\textbf{Mots-clés de référence}: humanities~research, \textcolor{color6}{digital}~\textcolor{color5}{libraries}, great~\textcolor{color2}{ideas}, hypertext, \textcolor{color8}{key}~phrases, \textcolor{color1}{quotations}, data~mining
    \vspace{0.2em}

    \end{tabular}%
    }
    \caption{Exemple de document de l'ensemble de test du corpus KP20k (id: {\small 012397})). Les mots composants termes-clés présent dans le document sont soulignés.}
\end{figure}
\fi
%\end{frame}

% explication absent présent et copie

\begin{frame}{Validation des performances des méthodes neuronales}

    \begin{block}{Problématique}
    Entraînement des méthodes génératives sur \textbf{un seul} jeu de données: \textbf{KP20k}.
    \end{block}
    
    %Pour ajouter un point de comparaison et valider les performances des méthodes génératives.
    Introduction de \textbf{KPTimes}, le seul jeu de données de \textbf{grande taille} d'\textbf{articles journalistiques} annotés en mots-clés par des \textbf{éditeurs}.
    
    \begin{block}{Conclusion}
    %\textbf{Analyse des performances des méthodes génératives:}
    \begin{enumerate}
        \item Résultats transposables à KPTimes.
        \item Faible généralisation à des documents de \textbf{genres différents} et à un \textbf{type d'annotation différent}.
        \item Faibles performances en partie liées à l'évaluation.
        %\item Sans données d'entraînement les \textbf{méthodes extractives} sont \textbf{compétitives} avec les méthodes neuronales.
    \end{enumerate}
    \end{block}

\end{frame}