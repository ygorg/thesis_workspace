%Heavily inspired from https://texample.net/tikz/examples/neural-network/

\begin{tikzpicture}[shorten >=1pt,->,draw=black!50]
    \def\myscale{1.5}
    \tikzstyle{neuron}=[circle,fill=black!25,minimum size=17*\myscale,inner sep=0cm]
    \tikzstyle{input neuron}=[neuron, fill=color1!60];
    \tikzstyle{hidden neuron}=[neuron, fill=color0!60];
    \tikzstyle{output neuron}=[neuron, fill=color2!60];
    \tikzstyle{edge label}=[above=-.025cm,sloped,scale=.4*\myscale,black!60]
    \tikzstyle{label}=[scale=.75*\myscale, text width=2cm, align=center]

    \def\layersep{2.5*\myscale}
    \def\ninput{2}
    \def\nlayerone{3}
    \def\nlayertwo{2}
    \def\noutput{2}

    % Draw the input layer nodes
    \foreach \name / \y in {1,...,\ninput}
    % This is the same as writing \foreach \name / \y in {1/1,2/2,3/3,4/4}
        \node[input neuron] (I-\name) at (0,-\y*\myscale) {};


    % Draw the hidden layer nodes
    \foreach \name / \y in {1,...,\nlayerone}
        \path[yshift=.5*\myscale cm]
            node[hidden neuron] (H1-\name) at (1*\layersep,-\y*\myscale) {};

    \foreach \name / \y in {1,...,\nlayertwo}
        \path[yshift=.0*\myscale cm]
            node[hidden neuron] (H2-\name) at (2*\layersep,-\y*\myscale) {};

    % Draw the output layer node
    \foreach \name / \y in {1,...,\noutput}
        \path[yshift=0*\myscale cm]
            node [output neuron] (O-\name) at (3*\layersep,-\y*\myscale) {};

    % Connect every node in the input layer with every node in the
    % hidden layer.
    
    %\foreach \source in {1,...,\ninput}
    %    \foreach \dest in {1,...,\nlayerone}
    %        \draw[->] (I-\source) -- node [edge label, pos={.28-}] {$W^1_{\source,\dest}$} (H1-\dest);
    
    \def\layerout{1}
    \def\source{1}
    \foreach \dest / \n in {1/.2,2/.25,3/.235}
        \draw[->] (I-\source) -- node [edge label, pos={\n}] {$W^\layerout_{\source,\dest}$} (H\layerout-\dest);
    \def\source{2}
    \foreach \dest / \n in {1/.17,2/.285,3/.26}
        \draw[->] (I-\source) -- node [edge label, pos={\n}] {$W^\layerout_{\source,\dest}$} (H\layerout-\dest);


    \def\layerout{2} % modify this
    \pgfmathtruncatemacro{\layerin}{\layerout - 1}
    \def\source{1} % modify this
    \foreach \dest / \n in {1/.2,2/.26} % modify this
        \draw[->] (H\layerin-\source) -- node [edge label, pos={\n}] {$W^\layerout_{\source,\dest}$} (H\layerout-\dest);
    \def\source{2} % modify this
    \foreach \dest / \n in {1/.2,2/.25} % modify this
        \draw[->] (H\layerin-\source) -- node [edge label, pos={\n}] {$W^\layerout_{\source,\dest}$} (H\layerout-\dest);
    \def\source{3} % modify this
    \foreach \dest / \n in {1/.23,2/.3} % modify this
        \draw[->] (H\layerin-\source) -- node [edge label, pos={\n}] {$W^\layerout_{\source,\dest}$} (H\layerout-\dest);

    % Connect every node in the hidden layer with the output layer
    \def\layerout{3} % modify this
    \pgfmathtruncatemacro{\layerin}{\layerout - 1}
    \foreach \source in {1,...,\nlayertwo}
        \foreach \dest in {1,...,\noutput}
            \draw[->] (H\layerin-\source) -- node [edge label, pos={.20+(1-Mod(\dest,2))*.05}] {$W^\layerout_{\source,\dest}$} (O-\dest);

    % Annotate the layers
    \node[label, below=.1cm of I-\ninput] (I-label) {Entrée};

    \node[draw, dashed, rounded corners,fill=none, fit=(H1-1) (H1-\nlayerone)] (H1) {};
    \node[label, below=.1cm of H1] (H1-label) {Première couche};

    \node[draw, dashed, rounded corners,fill=none, fit=(H2-1) (H2-\nlayertwo)] (H2) {};
    \node[label, below=.1cm of H2] (H2-label) {Seconde couche};

    \node[draw, dashed, rounded corners,fill=none, fit=(O-1) (O-\noutput)] (O) {};
    \node[label, below=.1cm of O] (O-label) {Couche de sortie};

    \node[draw, dashed, rounded corners,fill=none, fit=(H1) (H1-label) (H2) (H2-label)] (H) {};
    \node[label, text width=10cm, below=.1cm of H] (H-label) {Couches cachées};

\end{tikzpicture}