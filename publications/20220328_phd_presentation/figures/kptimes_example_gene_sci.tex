\begin{figure}
    \centering
    %\centerfloat
    \resizebox{0.98\textwidth}{!}{%
    \begin{tabular}{|p{1.3\textwidth}|}
    \textbf{\textcolor{color7}{Multiple} \textcolor{color0}{disorder} \textcolor{color1}{diagnosis} with adaptative \textcolor{color6}{competitive} \textcolor{color5}{neural} \textcolor{color8}{networks}}

    \vspace{0.9em}

    Backpropagation \textcolor{color5}{neural} \textcolor{color8}{networks} have repeatedly been used for \textcolor{color4}{diagnostic} \textcolor{color3}{problem-solving}, but have not been demonstrated to work well when \textcolor{color7}{multiple} \textcolor{color0}{disorders} are present.
    We hypothesized that letting nodes in a backpropagation \textcolor{color5}{neural} \textcolor{color8}{network} compete to be part of a \textcolor{color4}{diagnostic} solution would produce better performance than the use of existing backpropagation methods.
    %To test this hypothesis, we derived an error backpropagation \textcolor{color2}{learning} rule that can be used with \textcolor{color6}{competitive} units (\textcolor{color6}{competitive} backpropagation).
    %Artificial \textcolor{color5}{neural} \textcolor{color8}{networks} were then trained using both this new \textcolor{color2}{learning} rule and standard error backpropagation on a specific medical \textcolor{color1}{diagnosis} problem: identification of the location of damage in the brain given a set of examination findings.
    [...]
    %Training samples included solely 'prototypical' cases where a single location of damage is present.
    %The trained \textcolor{color8}{networks} were then tested with atypical cases where the manifestations of more than one \textcolor{color0}{disorder} were present or only a single manifestation was present.
    %\textcolor{color8}{Networks} employing \textcolor{color6}{competition} among units were found to perform qualitatively better with these multiple-disorder cases than standard \textcolor{color8}{networks} and also to perform better on single-manifestation cases.
    %The reasons for this are explained.
    %The \textcolor{color6}{competitive} backpropagation \textcolor{color2}{learning} rule described here provides a promising new tool for adaptive \textcolor{color4}{diagnostic} \textcolor{color3}{problem-solving}.

    \vspace{1.1em}

    \textbf{CopyRNN}: \textcolor{color5}{competitive}~\textcolor{color1}{backpropagation}, \textcolor{color1}{backpropagation}, \textcolor{color1}{adaptive}, \textcolor{color8}{neural}~\textcolor{color2}{networks}, \textcolor{color5}{competitive}~\textcolor{color0}{learning}\\

    \textbf{CopyNews}: research, science~and~technology, medicine~and~health, \textcolor{color4}{diagnostic}~\textcolor{color3}{problem-solving}, science~journal\\
    \textbf{Réf. auteur}: \textcolor{color1}{diagnosis}, \textcolor{color7}{multiple}~\textcolor{color0}{disorders}, \textcolor{color6}{competition}, \textcolor{color5}{neural}~\textcolor{color8}{networks}, \textcolor{color2}{learning}
    \vspace{0.2em}

    \end{tabular}%
    }
    %\caption{Exemple de document de l'ensemble de test du corpus KP20k (id: {\small 011355}))}%. Les mots composants termes-clés présent dans le document sont soulignés.}
\end{figure}


\iffalse
\begin{figure}
    \centering
    %\centerfloat
    \resizebox{0.98\textwidth}{!}{%
    \begin{tabular}{|p{1.3\textwidth}|}
    \textbf{Ex-Nissan chief Carlos Ghosn to be served with fresh arrest warrant }

    \vspace{0.9em}

    Tokyo prosecutors have decided to seek a fresh arrest warrant for former Nissan Motor Co. Chairman Carlos Ghosn on suspicion that he failed to report around ¥4 billion (\$35.5 million) of his remuneration in its securities reports for the three years through March, sources close to the matter said Tuesday.
    %Ghosn, who is being held at the Tokyo Detention House, has already been accused of breaching the Financial Instruments and Exchange Act after allegedly reporting only ¥5 billion of his ¥10 billion compensation during the five years through March 2015.
    %Along with Greg Kelly, a former Nissan representative director, Ghosn is expected to be served with a fresh arrest warrant on Dec. 10, when the detention period for the pair expires.
    %Ghosn and Kelly, who was arrested along with the former chairman on Nov. 19 for alleged conspiracy, have told the prosecutors that it was unnecessary to report some of the remuneration as the payments had yet to be settled, according to different sources with knowledge of the investigation.
    %The unreported remuneration of the 64-year-old charismatic automotive industry figure, known for rescuing Nissan from the brink of bankruptcy in the 1990s, is believed to total ¥9 billion.
    %In Japan, crime suspects can be kept in custody for 10 days and that can be extended for another 10 days if a judge grants prosecutors’ request for extension.
    %At the end of that period, prosecutors must file a former charge or let the suspect go.
    %However, they can also arrest suspects for a separate crime, in which case the process starts over again.
    %This process can be repeated, sometimes keeping suspects detained for months without formal charges and without bail.

    \vspace{1.1em}

    \textbf{CopyRNN}: fresh arrest warrant, ghosn, fresh arrest, ex-nissan, bankruptcy\\
    \textbf{CopyNews}: ghosn carlos, carlos ghosn, japan, tokyo detention house, billion compensation\\
    \textbf{Mots-clés de référence}: arrest, nissan, mitsubishi motors, renault, carlos ghosn
    \vspace{0.2em}

    \end{tabular}%
    }
    %\caption{Exemple de document de l'ensemble de test du corpus KPTimes (id: {\small jp0010292})).}
    \label{fig:_}
\end{figure}





\begin{figure}
    \centering
    %\centerfloat
    \resizebox{0.98\textwidth}{!}{%
    \begin{tabular}{|p{1.3\textwidth}|}
    \textbf{Exploring a \textcolor{color6}{digital} \textcolor{color5}{library} through \textcolor{color8}{key} \textcolor{color2}{ideas}.}

    \vspace{0.9em}

    \textcolor{color8}{Key} \textcolor{color2}{Ideas} is a technique for exploring \textcolor{color6}{digital} \textcolor{color5}{libraries} by navigating passages that repeat across multiple \textcolor{color3}{books}.
    From these popular passages emerge \textcolor{color1}{quotations} that authors have copied from \textcolor{color3}{book} to \textcolor{color3}{book} because they capture an \textcolor{color2}{idea} particularly well: \textcolor{color0}{Jefferson} on liberty; \textcolor{color4}{Stanton} on women's rights; and Gibson on cyberpunk.
    We augment Popular Passages by extracting \textcolor{color8}{key} terms from the surrounding context and computing sets of related \textcolor{color8}{key} terms.
    We then create an interaction model where readers fluidly explore the \textcolor{color5}{library} by viewing popular \textcolor{color1}{quotations} on a particular \textcolor{color8}{key} term, and follow links to \textcolor{color1}{quotations} on related \textcolor{color8}{key} terms.
    %In this paper we describe our vision and motivation for \textcolor{color8}{Key} \textcolor{color2}{Ideas}, present an implementation running over a massive, real-world \textcolor{color6}{digital} \textcolor{color5}{library} consisting of over a million scanned \textcolor{color3}{books}, and describe some of the technical and design challenges. The principal contribution of this paper is the interaction model and prototype system for browsing \textcolor{color6}{digital} \textcolor{color5}{libraries} of \textcolor{color3}{books} using \textcolor{color8}{key} terms extracted from the aggregate context of popularly quoted passages.

    \vspace{1.1em}

    \textbf{CopyNews}: \textcolor{color3}{books}, \textcolor{color4}{stanton}, \textcolor{color0}{jefferson}, art, \textcolor{color5}{library}\\
\textbf{Mots-clés de référence}: humanities~research, \textcolor{color6}{digital}~\textcolor{color5}{libraries}, great~\textcolor{color2}{ideas}, hypertext, \textcolor{color8}{key}~phrases, \textcolor{color1}{quotations}, data~mining
    \vspace{0.2em}

    \end{tabular}%
    }
    \caption{Exemple de document de l'ensemble de test du corpus KP20k (id: {\small 012397})). Les mots composants termes-clés présent dans le document sont soulignés.}
\end{figure}
\fi