\section*{Conclusion}

\begin{frame}{}
    \sectionpage
\end{frame}

\begin{frame}{Analyse de l'existant}
    \begin{itemize}
        \item \textbf{\'Evaluation stricte} des méthodes de l'état de l'art réalisée grâce à la création du jeu de données \textbf{KPTimes}.
        \item Comparaison directe des méthodes impossible avant cette étude.
    \end{itemize}
    \begin{itemize}
        \item Méthodes génératives peu généralisables à d'autres genres de documents et annotation.
        \item Sans données d'entraînement les \textbf{méthodes de base} sont toujours compétitives.
        \item Avec données d'entraînement les \textbf{méthodes génératives} sont l'état de l'art.
        \item \'Evaluation par \textbf{appariement exact} à une référence peu fiable. Confirme l'évaluation manuelle réalisée par  \citet{bougouin_indexation_2015}.
    \end{itemize}

\end{frame}

\begin{frame}{Traiter les faiblesses de l'évaluation automatique}
    \begin{itemize}
        \item \textbf{\'Evaluation extrinsèque} pour étudier la qualité des mots-clés dans un \textbf{cadre applicatif}.
        %\item Premiers travaux 
        %\item Nos travaux sont les premiers à montrer l'intérêt des mots-clés produits automatiquement dans une tâche applicative.

        \item Mots-clés produits par les récentes méthodes génératives sont \textbf{assez qualitatifs} pour améliorer une tâche de recherche documentaire contrairement aux méthodes extractives.
    
    \end{itemize}
\end{frame}

\section*{Perspectives}

\begin{frame}{}
    \sectionpage
\end{frame}

\begin{frame}{Perspectives à court terme}
    \begin{itemize}
    \item \textbf{Évaluer les méthodes génératives plus récentes} pour mesurer l'impact de leurs améliorations incrémentales sur la recherche documentaire.

    \item \textbf{Jeux de données en français} pour transposer nos expériences à cette langue. Peu pertinent pour l'informatique mais pertinent pour les sciences sociales par exemple.
    
    \item \textbf{Cohérence des mots-clés produits} pour aider à la navigation des bibliothèques.
    
    %\item \textbf{Multiplier les tâches applicatives} pour l'évaluation extrinsèque. Création d'un benchmark tel que GLUE pour faciliter la comparaison des méthodes.
    \end{itemize}
\end{frame}

\begin{frame}{Perspectives à long terme}
    \begin{itemize}
    \item Les mots-clés sont considérés comme une \textbf{fin en soi}.
    %\item Mots-clés non pertinent ou usage non adapté ?
    \item \textbf{Redéfinition de la tâche}: mots-clés pour la navigation, pour la RI, pour la catégorisation, pour la création de thésaurus, \ldots
    %\item Peu de recherche/outils sur la \textbf{navigation} assisté par des mots-clés (\citet{jones_phrasier_1999}, Microsoft Academics, AMiner).
    
    \item Les mots-clés sont assez qualitatifs, comment les intégrer aux bibliothèques, systèmes de RI ?
    %\item Pallier la qualité des mots-clés auteurs:
    %\item \textbf{Diversifier les sources d'annotation} pour ne pas reposer sur les mots-clés auteurs.
    %\item \textbf{Génération de mots-clés non supervisée} pour ne plus être limité par les mots-clés auteurs.
    \end{itemize}
    %\textbf{Disponibilité d'annotation qualitatives}: les méthodes prometteuses nécessitent de grandes quantités de données annotés mais la grande majorité des annotations en mots-clés sont peu qualitatives

    %\textbf{Lien avec la RI}: évaluer l'apport des mots-clés par rapport aux ré-ordonnanceurs et le passage à l'échelle des m
\end{frame}