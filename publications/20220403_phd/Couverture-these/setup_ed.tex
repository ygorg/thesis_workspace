\usepackage{multicol} % pour le résumé
\usepackage{tikz} % tikz est utilise pour tracer des boites, par exemple
\usepackage{wallpaper}
\usepackage{geometry}
\usepackage{setspace} % permet d'utiliser les commandes \spacing, doublespace (double interligne), singlespace (simple interligne) et onehalfspace (un interligne et demi)
\usepackage{fancyhdr} % Afin de r\'{e}aliser soi-même les en-têtes et pieds de page, voir chaque d\'{e}but de chapitre.

%%%%%%%%%%%%%%%%%%%%%%%%%%%%%%%%%%%%%%%%%%%%%%%%
%%%%%%%%%%%%%%%% EN-TETES PAGES %%%%%%%%%%%%%%%%

%%%%%%%%% Pour supprimer les en-têtes et pied de page gênants par exemple juste avant un chapitre sur une page de droite
\newcommand{\clearemptydoublepage}{%
   \newpage{\pagestyle{empty}\cleardoublepage}}
%%%% .... et utiliser la commande \clearemptydoublepage juste avant le \chapter

\fancyhf{}                       % on annule le fancy automatique

%%%%%%%%%% Gerer les en-tètes dans les frontmatter mainmatter et backmatter


\appto\frontmatter{\pagestyle{fancy}
\renewcommand{\sectionmark}[1]{}
\renewcommand{\chaptermark}[1]{\markboth{\textit{#1}}{}}
\fancyhead[RE,LO]{\small\thepage}
    \fancyhead[RO]{\small\leftmark} % \rightmark = section courante
    \fancyhead[LE]{\small\leftmark} % \leftmark = chapitre courant
    %\fancyfoot[C]{\thepage}               % marque la page au centre
}

\appto\mainmatter{\pagestyle{fancy}
\renewcommand{\sectionmark}[1]{\markright{\textit{\thesection.\ #1}}}
\renewcommand{\chaptermark}[1]{\markboth{\chaptername~\thechapter~--\ \textit{#1}}{}}
\fancyhead[RE,LO]{\small\thepage}
    \fancyhead[RO]{\small\rightmark} % \rightmark = section courante
    \fancyhead[LE]{\small\leftmark}  % \leftmark = chapitre courant
    %\fancyfoot[C]{\thepage}               % marque la page au centre
}

\appto\backmatter{\pagestyle{fancy}
\renewcommand{\sectionmark}[1]{\markright{\thesection.\ #1}}
\renewcommand{\chaptermark}[1]{\markboth{\chaptername~\thechapter~--\ #1}{}}
\fancyhead[RE,LO]{\small\thepage}
    \fancyhead[RO]{}   % \rightmark = section courante
    \fancyhead[LE]{} % \leftmark = chapitre courant
    %\fancyfoot[C]{\thepage}               % marque la page au centre
}



%%%%%%%%%%%%%%%%%%%%%%%%%%%%%%%%%%%%%%%%%%%%%%
%%%%%%%%%%%%%%%% En-tete chap %%%%%%%%%%%%%%%%

\newcommand*{\selectfontchapheads}{\fontfamily{phv}\selectfont} % Font style used chapter headings

\makeatletter
\def\thickhrulefill{\leavevmode \leaders \hrule height 1ex \hfill \kern \z@}
\def\@makechapterhead#1{%
  \vspace*{-30\p@}%
  {\parindent \z@ \raggedleft \reset@font
            \scshape \@chapapp{} \thechapter
        \par\nobreak
        \interlinepenalty\@M
    \selectfontchapheads \Huge \bfseries #1\par\nobreak
    %\vspace*{1\p@}%
    \hrulefill
    \par\nobreak
    \vskip 50\p@
  }}
\def\@makeschapterhead#1{%
 \vspace*{-50\p@}%
  {\parindent \z@ \raggedleft \reset@font
            \scshape \vphantom{\@chapapp{} \thechapter}
        \par\nobreak
        \interlinepenalty\@M
    \selectfontchapheads \Huge \bfseries #1 \par\nobreak
    %\vspace*{1\p@}%
    \hrulefill
    \par\nobreak
    \vskip 30\p@
  }}

% To make Table of content look like a chapter
\def\@cftmaketoctitle{%
    \chapter*{\contentsname}
}
\makeatother


\newcommand\hauteurlogos[3]{
    \hauteurlogoecole{#1}
    \hauteurlogoetablissementA{#2}
    \hauteurlogoetablissementB{#3}
}

% Define commands to set fonts throughout the document
\newcommand*{\selectfontfrontcover}{\fontfamily{phv}\selectfont}  % Font style used in front cover 
\newcommand*{\selectfontbackcover}{\fontfamily{phv}\selectfont}   % Font style used in back cover 
%%%%%%%%%%%%%%%%%%%%%%%%%%%%%%%%%%%%%%%%%%%%%%%%%%%%%%%%%
%%%%%%%%%%%%%%%% VARIABLES PAGE DE GARDE %%%%%%%%%%%%%%%%

%%%%% Dossier contenant les info de l'ecole doctorale
\newcommand*{\direcole}[1]{\gdef\vdirecole{Couverture-these/#1}}
\direcole{}

\makeatletter
%%%%% Nom ecole, une variable par ligne
\newcommand{\nomecoleA}[1]{\gdef\@nomecoleA{#1}}
\nomecoleA{}
\newcommand{\nomecoleB}[1]{\gdef\@nomecoleB{#1}}
\nomecoleB{}

%%%%% Numero ecole doctorale
\newcommand{\numeroecole}[1]{\gdef\@numeroecole{#1}}
\numeroecole{}
\makeatother

%%%% Etablissement delivrant le diplome, une variable par ligne
\newcommand{\nometablissementA}[1]{\gdef\vnometablissementA{#1}}
\nometablissementA{}
\newcommand{\nometablissementB}[1]{\gdef\vnometablissementB{#1}}
\nometablissementB{}
\newcommand{\nometablissementC}[1]{\gdef\vnometablissementC{#1}}
\nometablissementC{}
\newcommand{\nometablissementD}[1]{\gdef\vnometablissementD{#1}}
\nometablissementD{}
\newcommand{\nometablissementE}[1]{\gdef\vnometablissementE{#1}}
\nometablissementE{}

%%%% Logos etablissement delivrant le diplome, supporte deuble affiliation
\newcommand*{\logoetablissementA}[1]{\gdef\vlogoetablissementA{#1}}
\logoetablissementA{}
\newcommand*{\logoetablissementB}[1]{\gdef\vlogoetablissementB{#1}}
\logoetablissementB{}

%%%% Hauteur des logos, variable selon les (double) affiliations
\newcommand*{\hauteurlogoecole}[1]{\gdef\vhauteurlogoecole{#1}}
\hauteurlogoecole{2.4cm}
\newcommand*{\hauteurlogoetablissementA}[1]{\gdef\vhauteurlogoetablissementA{#1}}
\hauteurlogoetablissementA{}
\newcommand*{\hauteurlogoetablissementB}[1]{\gdef\vhauteurlogoetablissementB{#1}}
\hauteurlogoetablissementB{2.4cm}

\makeatletter
%%%% Eventuel sous-titre
\newcommand{\lesoustitre}[1]{\gdef\@lesoustitre{#1}}
\lesoustitre{}

%%%% Discipline
\newcommand{\discipline}[1]{\gdef\@discipline{#1}}
\discipline{}

%%%% Jury
\newcommand{\jury}[1]{\gdef\@jury{#1}}
\jury{}

%%%%% Sp\'{e}cialit\'{e}
\newcommand{\spec}[1]{\gdef\@spec{#1}}
\spec{}

%%% Ville de soutenance
\newcommand{\lieu}[1]{\gdef\@lieu{#1}}
\lieu{}

%%% Unite de recherche: laboratoire / department / unit\'{e}
\newcommand{\uniterecherche}[1]{\gdef\@uniterecherche{#1}}
\uniterecherche{}

%%% Num\'{e}ro de la th\`{e}se
\newcommand{\numthese}[1]{\gdef\@numthese{#1}}
\numthese{}
\makeatother


%%%%%%%%%%%%%%%%%%%%%%%%%%%%%%%%%%%%%%%%%%%%%%%
%%%%%%%%%%%%%%%% PAGE DE GARDE %%%%%%%%%%%%%%%%

% Define some font sizes specific to the covers, supposed to be in 12pt
\newcommand{\HugeTwelve}{\fontsize{26}{31}\selectfont} % 12pt \Huge
\newcommand{\LARGETwelve}{\fontsize{20.74}{25}\selectfont} % 12pt \LARGE
\newcommand{\LargeTwelve}{\fontsize{16}{19}\selectfont} % 12pt \Large
\newcommand{\largeTwelve}{\fontsize{14.4}{17}\selectfont} % 12pt \large
\newcommand{\normalTwelve}{\fontsize{12}{13.2}\selectfont} % 12pt \normalsize
\newcommand{\smallTwelve}{\fontsize{11}{13.5}\selectfont} % 12pt \small
\newcommand{\footnotesizeTwelve}{\fontsize{9.5}{11}\selectfont} % 12pt \footnotesize

% Affiche les logos sur les pages de couverture
\newcommand{\displayLogos}{%
  \thispagestyle{empty}
  \begin{tikzpicture}[remember picture,overlay,line width=0mm]
    \node[xshift=6.2cm,yshift=2cm] {%
    \parbox{\textwidth}{%
      % Quand UR1 est l'unique etablissement, il ne faut afficher que son logo
      {\ifthenelse{\equal{\vlogoetablissementA}{}\and\equal{\vlogoetablissementB}{UR1-noir}}{
        $\vcenter{\hbox{%
          \includegraphics[keepaspectratio,height=\vhauteurlogoetablissementB,width=7cm
          ]{Couverture-these\vlogoetablissementB}%
        }}$
      }{%
        $\vcenter{\hbox{%
          \includegraphics[keepaspectratio,height=\vhauteurlogoecole,%width=7cm
          ]{\vdirecole/logo}%
        }}$
        \hfill
        {\if\vlogoetablissementA\empty \else
          $\vcenter{\hbox{%
            \includegraphics[keepaspectratio,height=\vhauteurlogoetablissementA,width=7cm
            ]{Couverture-these\vlogoetablissementA}%
          }}$
        \fi}%
        \hspace{3mm}
        $\vcenter{\hbox{%
          \includegraphics[keepaspectratio,height=\vhauteurlogoetablissementB,width=7cm
          ]{Couverture-these/\vlogoetablissementB}%
        }}$
      }}%
    }%
  };
  \end{tikzpicture}
  \par\nobreak
}

%mise en page de la page de garde
\makeatletter
\def\maketitle{%
  \newgeometry{inner=30mm,outer=20mm,top=40mm,bottom=20mm}
  \onehalfspacing
  \thispagestyle{empty}
  \clearpage
  %background image of the front cover
  \AddToShipoutPicture*{%
    \put(0,0){%
    \parbox[b][42.6cm]{\paperwidth}{%
        \vfill
        \includegraphics[width=\paperwidth,keepaspectratio,trim={0 5pt 0 0}]{\vdirecole/image-fond-garde} % Must trim white border off of bottom
        \begin{tikzpicture}
            \fill[color-ecole] (0,0) rectangle (\paperwidth,4.4);
        \end{tikzpicture}
        \vfill
  }}}
  \displayLogos
  %
  \begin{tikzpicture}[remember picture,overlay,line width=0mm]
   \node at (current page.center)
{\parbox{17.6cm}{
\vspace{3.6cm}

\selectfontfrontcover % Set font style for front cover page

{\HugeTwelve \textsc{Th\`{e}se de doctorat de} \\}

% \vspace{5mm}
{\normalTwelve \if\@nomecoleB\empty ~\\ \else \fi} % To compensate the 2 lines of MathSTIC
{\setlength{\baselineskip}{0.9\baselineskip}
{\largeTwelve \if\vnometablissementA\empty ~ \else \vnometablissementA \fi} \\
{\largeTwelve \if\vnometablissementB\empty ~ \else \vnometablissementB \fi} \\
{\largeTwelve \if\vnometablissementC\empty ~ \else \vnometablissementC \fi} \\
{\largeTwelve \if\vnometablissementD\empty ~ \else \vnometablissementD \fi} \\
{\largeTwelve \vnometablissementE} \\
\par}
\vspace{0.01cm}
{\setlength{\baselineskip}{0.7\baselineskip}
{\smallTwelve \textsc{\'{E}cole Doctorale \No \@numeroecole}} \\
{\normalTwelve \textit{\@nomecoleA}} \\
{\normalTwelve \if\@nomecoleB\empty \else \textit{\@nomecoleB} \\ \fi}
{\normalTwelve Sp\'{e}cialit\'{e} : \textit{\@spec}}

%\fontsize{12}{10}\selectfont
\vspace{0.5cm}
\hspace{0.6cm}{\normalTwelve Par \vspace{0.15cm}}
\par}
\hspace{0.6cm}{\LARGETwelve \textbf{\@author}} \vspace{0.5cm}

{\LargeTwelve \textbf{\@title}} \vspace{0.5cm}
  
{\largeTwelve \@lesoustitre} \vspace{0.5cm}
\begin{spacing}{1}
   \smallTwelve
   \textbf{Th\`{e}se pr\'{e}sent\'{e}e et soutenue \`{a} \@lieu, le \@date} \\
   \textbf{Unit\'{e} de recherche : \@uniterecherche} \\
   \textbf{\if\@numthese\empty \else Th\`{e}se \No : \@numthese \fi} % Hide line if no number provided
\end{spacing}
\vspace{1.3cm}
  \begin{small}
  \begin{spacing}{1}
     \@jury
  \end{spacing}
  \end{small}
}
};
\end{tikzpicture}
  \restoregeometry
}

\makeatother



%%%%%%%%%%%%%%%%%%%%%%%%%%%%%%%%%%%%%%%%%%%%%%%%%%%%%%%%%
%%%%%%%%%%%%%%%% QUATRIEME DE COUVERTURE %%%%%%%%%%%%%%%%

\newcommand{\backcoverheader}{%
\thispagestyle{empty}
\AddToShipoutPicture*{%
    \put(0,0){%
    \parbox[t][\paperheight]{\paperwidth}{%
        \vspace{-29.6cm}
        \includegraphics[width=\paperwidth,height=\paperheight,keepaspectratio]{\vdirecole/image-fond-dos}%
    }}
    \put(0,0){%
    \parbox[t][\paperheight]{\paperwidth}{%
        \vspace{-14.5cm}
        \includegraphics[width=\paperwidth,height=\paperheight,keepaspectratio]{\vdirecole/image-fond-dos2}%
    }}
}
\hspace{9mm}
\displayLogos
}

\newcommand{\titleFR}[1]{%
\vspace{1cm}
{\centering \noindent \textcolor{color-ecole}{\rule{\textwidth}{0.2cm}}}
\vspace{-1cm}
\selectlanguage{french}
\section*{\selectfontbackcover\smallTwelve \textcolor{color-ecole}{Titre : }{\selectfontbackcover\mdseries{#1}}} % In this particular case, font style needs to get re-selected locally
}

\newcommand{\keywordsFR}[1]{%
\vspace{-0.2cm}
\noindent{\smallTwelve \textbf{Mot cl\'{e}s : }#1}
}

\newcommand{\abstractFR}[1]{%
\vspace{-0.2cm}
\begin{multicols}{2}
\begin{spacing}{1}
	\noindent\smallTwelve \textbf{R\'{e}sum\'{e} : }#1
\end{spacing}
\end{multicols}
}

\newcommand{\titleEN}[1]{%
\vspace{0.5cm}
{\centering \noindent \textcolor{color-ecole}{\rule{\textwidth}{0.2cm}}}
\vspace{-1cm}
\selectlanguage{english}
\section*{\selectfontbackcover\smallTwelve \textcolor{color-ecole}{Title: }{\selectfontbackcover\mdseries{#1}}} % In this particular case, font style needs to get re-selected locally
}

\newcommand{\keywordsEN}[1]{%
\vspace{-0.2cm}
\noindent{\smallTwelve \textbf{Keywords: }#1}
}

\newcommand{\abstractEN}[1]{%
\vspace{-0.2cm}
\begin{multicols}{2}
\begin{spacing}{1}
	\noindent\smallTwelve \textbf{Abstract: }#1
\end{spacing}
\end{multicols}
}